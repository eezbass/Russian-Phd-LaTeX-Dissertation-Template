\section{Списки}
\begin{frame}[plain, noframenumbering]
    \begin{center}
        \Huge
        Списки
    \end{center}
\end{frame}

\subsection{Нумерованные}

\begin{frame}
    \frametitle{Нумерованные списки}
    \begin{enumerate}
        \item один
        \item два
        \item три
    \end{enumerate}
\end{frame}
\note{
    Этот текст будет виден только если его отображение включено в файле \textbf{Presentation/setup}.
    Для раздельного вывода презентации и заметок на разные экраны (как в impress или powerpoint) можно использовать программу \textit{pdf-presenter-console}.
}

\subsection{Не нумерованные}


\begin{frame}
    \frametitle{Перечисления}
    \begin{itemize}
        \item Проблема 1
        \item Проблема 2
        \item Проблема 3
    \end{itemize}
\end{frame}
\note[itemize]{
    \item Тезис 1
    \item Тезис 2
    \item Тезис 3
}

\subsection{Комбинированные}

\begin{frame}
    \frametitle{Комбинация списков}
    \begin{enumerate}
        \item \textbf{Задача 1}
              \begin{itemize}
                  \item Подзадача 1-1
                  \item Подзадача 1-2
              \end{itemize}
        \item \textbf{Задача 2}
              \begin{itemize}
                  \item Подзадача 2-1
                  \item Подзадача 2-2
                  \item Подзадача 2-3
              \end{itemize}
        \item \textbf{Задача 3}
              \begin{itemize}
                  \item Подзадача 3-1
                  \item Подзадача 3-2
                  \item Подзадача 3-3
              \end{itemize}
    \end{enumerate}
\end{frame}
\note[itemize]{
    \item Задача 1
    \item Задача 2
    \item Задача 3
}

\begin{frame}[allowframebreaks]
    \frametitle{Разделение слайда}
    Поясняющий текст
    \begin{itemize}
        \item Один
        \item Два
        \item Три
    \end{itemize}
    \framebreak
    Продолжение предыдущего слайда
\end{frame}

\section{Графика}
\begin{frame}[plain, noframenumbering]
    \begin{center}
        \Huge
        Графика
    \end{center}
\end{frame}


\begin{frame}
    \frametitle{Одиночное изображение}
    \centering
    \includegraphics[width=0.8\linewidth]{latex} % окружение figure не требуется
\end{frame}

\begin{frame}
    \frametitle{Векторная графика}
    \begin{figure}
	    \centering
	    \ifdefmacro{\tikzsetnextfilename}{\tikzsetnextfilename{tikz_presentation}}{}% присваиваемое предкомпилированному pdf имя файла (не обязательно)
	    \begin{tikzpicture}
  \draw[<->,thick] (0,6) -- (0,0) -- (10,0); % axis
  \draw[help lines] (3,0) -- (3,5); % fill end
  \draw[help lines] (6,0) -- (6,5); % flattop end
  \draw[help lines] (9,0) -- (9,5); % decay end
  \draw[<->, thin] (0,5) -- (3,5); % fill
  \draw[<->, thin] (3,5) -- (6,5); % flattop
  \draw[<->, thin] (6,5) -- (9,5); % decay
  \draw[blue, ultra thick] (0,0) -- (0,4) -- (6,4) -- (6,0); % power
  \draw[red, ultra thick, domain=0:3] plot (\x, {4*(1 - exp(-1.5*\x))}); % fill plot
  \draw[red, ultra thick, domain=3:6] plot (\x, {4*(1 - exp(-1.5*\x))}); % flattop plot
  \draw[red, ultra thick, domain=6:9.5] plot (\x, {4*(1 - exp(-9)) - 4*(1 - exp(-1.5*\x+9))}); % decay plot
  \node [above] at (0,6) {\(E_{acc}\)};
  \node [right] at (10,0) {\(t\)};
  \node at (1.5,5.4) {заполнение};
  \node at (4.5,5.4) {работа};
  \node at (7.5,5.4) {затухание};
  \node at (4.5,1.4) {пучки};
  \foreach \x in {3.1,3.3,...,5.9} {
    \draw[->, green, thick] (\x,0) -- (\x,1);
  }
\end{tikzpicture}
    \end{figure}
\end{frame}

\subsection{Расположение}

\begin{frame}
    \frametitle{Изображения по-вертикали}
    \centering
    \vfill
    \includegraphics[width=0.8\linewidth,height=0.1\textheight]{latex} \\
    \TeX
    \vfill
    \includegraphics[width=0.8\linewidth,height=0.2\textheight]{latex} \\
    \LaTeX
    \vfill
    \includegraphics[scale=0.2]{latex} \\
    \vfill
\end{frame}


\begin{frame}
    \frametitle{Изображения по-горизонтали}
    \begin{minipage}[t]{0.47\linewidth}
        \textbf{Составная \\ подпись 1}
        \center{\includegraphics[width=1\linewidth]{knuth1}}
    \end{minipage}
    \hfill
    \begin{minipage}[t]{0.47\linewidth}
        \textbf{Составная \\ подпись 2}
        \center{\includegraphics[width=1\linewidth]{knuth2}}
    \end{minipage}
\end{frame}

\subsection{Линии}

\begin{frame}
    \frametitle{Разделяющие линии}
    \begin{minipage}[c]{0.47\linewidth}
        \center{\includegraphics[width=1\linewidth]{latex}}
        \bigskip
        \hrule{}
        \bigskip
        \textbf{Составная \\ подпись 1}
    \end{minipage}
    \hfill
    \vrule{}
    \hfill
    \begin{minipage}[c]{0.47\linewidth}
        \flushright
        \textbf{Составная \\ подпись 2}
        \center{\includegraphics[width=1\linewidth]{knuth2}}
    \end{minipage}
\end{frame}

\section{Остальное}
\begin{frame}[plain, noframenumbering]
    \begin{center}
        \Huge
        Остальное
    \end{center}
\end{frame}

\subsection{Формулы}

\begin{frame}
    \frametitle{Формулы}
    \[
    \left\{
    \begin{array}{rl}
        \dot x = & \sigma (y-x)  \\
        \dot y = & x (r - z) - y \\
        \dot z = & xy - bz
    \end{array}
    \right.
    \]
\end{frame}

\begin{frame}
    \frametitle{amsmath}
    \centering
    \begin{minipage}[t]{0.5\linewidth}
        \begin{multline*}
            y = 1 x^1 + 2 x^2 + 3 x^3 + \\ + 4 x^4 + 5 x^5 + \dots
        \end{multline*}
    \end{minipage}
\end{frame}

\begin{frame}[allowframebreaks]
    \frametitle{Уравнения Максвелла}
    \centering{
        \small
        \def\arraystretch{1.8}%
        \begin{tabular}{ll}
            \toprule
            Интегральная форма                                                                                                                                          & Дифференциальная форма                                                        \\ \midrule
            \(Q_e(t) = \displaystyle\oiint_S \vec D(t) \cdot d\vec{s} = \displaystyle\iiint_V \rho_v(t) dv\)                                                              & \(\nabla \cdot \vec D(t) = \rho_v(t)\)                                          \\
            \(\displaystyle\oiint_S \vec B(t) \cdot d\vec{s} = 0\)                                                                                                        & \(\nabla \cdot \vec B(t) = 0\)                                                  \\
            \(V_{emf}(t) = \displaystyle\oint_L \vec E(t) \cdot d\vec{l}\) = \(- \displaystyle\iint_S \left[\frac{\partial\vec{B}(t)}{\partial t}\right] \cdot d\vec{s}\)   & \(\nabla \times \vec E(t) = - \frac{\partial\vec{B}(t)}{\partial t}\)           \\
            \(I(t) = \displaystyle\oint_L \vec H(t) \cdot d\vec{l} = \displaystyle\iint_S \left[\vec J(t) + \frac{\partial\vec{D}(t)}{\partial t}\right] \cdot d\vec{s}\) & \(\nabla \times \vec H(t) = \vec J(t) + \frac{\partial\vec{D}(t)}{\partial t}\) \\ \midrule
            \(\displaystyle\oiint_S \vec J \cdot d\vec{s} = -\frac{\partial Q_e}{\partial t}\)                                                                            & \(\nabla \cdot \vec J = - \frac{\partial \rho_v}{\partial t}\)                  \\
            \bottomrule
            \multicolumn{2}{c}{\(\vec D(t) = \left[\varepsilon(t)\right] * \vec E(t)\)}                                                                                                                                                                   \\
            \multicolumn{2}{c}{\(\vec B(t) = \left[\mu(t)\right] * \vec H(t)\)}                                                                                                                                                                           \\
        \end{tabular}
    }
    \framebreak

    \hspace{0.05\linewidth}
    \centering{
        \small
        \def\arraystretch{1.8}%
        \begin{tabular}{ll}
            \toprule
            Интегральная форма                                                                                                            & Дифференциальная форма                             \\ \midrule
            \(Q_e = \displaystyle\oiint_S \vec D \cdot d\vec{s} = \displaystyle\iiint_V \rho_v dv\)                                         & \(\nabla \cdot \vec D = \rho_v\)                     \\
            \(\displaystyle\oiint_S \vec B \cdot d\vec{s} = 0\)                                                                             & \(\nabla \cdot \vec B = 0\)                          \\
            \(V_{emf} = \displaystyle\oint_L \vec E \cdot d\vec{l}\) = \(- \displaystyle\iint_S \left[j \omega \vec B\right] \cdot d\vec{s}\) & \(\nabla \times \vec E = - j \omega \vec B\)         \\
            \(I = \displaystyle\oint_L \vec H \cdot d\vec{l} = \displaystyle\iint_S \left[\vec J + j \omega \vec D\right] \cdot d\vec{s}\)  & \(\nabla \times \vec H = \vec J + j \omega \vec{D}\) \\ \midrule
            \(\displaystyle\oiint_S \vec J \cdot d\vec{s} = - j \omega Q_e\)                                                                & \(\nabla \cdot \vec J = - j \omega \rho_v\)          \\
            \bottomrule
            \multicolumn{2}{c}{\(\vec D(t) = \left[\varepsilon\right] \vec E(t)\)}                                                                                                               \\
            \multicolumn{2}{c}{\(\vec B(t) = \left[\mu\right] \vec H(t)\)}                                                                                                                       \\
        \end{tabular}
    }
\end{frame}

\subsection{Таблицы}

\begin{frame}
    \frametitle{Таблица}
    \centering
    \begin{tabular}{|l|l|}
        \hline
        \textbf{Заголовок 1} & \textbf{Заголовок 2} \\
        \hline
        Сумма                & \(b+a\)                \\
        \hline
        Разность             & \(a-b\)                \\
        \hline
        Произведение         & \(a*b\)                \\
        \hline
    \end{tabular}
\end{frame}

\begin{frame}
    \frametitle{Другая таблица}
    \centering
    \begin{tabular}{lc}
        \toprule
        \multicolumn{1}{c}{\textbf{Заголовок 1}} & \textbf{Заголовок 2} \\ \midrule
        Сумма                                    & \(b+a\)                \\
        Разность                                 & \(a-b\)                \\
        Произведение                             & \(a*b\)                \\
        \bottomrule
    \end{tabular}
\end{frame}


\subsection{Разное}

\begin{frame}
    \frametitle{Большой многоуровневый список}
    \begin{itemize}
        \item \textbf{Пункт 1}
              \begin{itemize}
                  \itemi Подпункт 1-1
                  \itemi Подпункт 1-2
              \end{itemize}
        \item \textbf{Пункт 2}
              \begin{itemize}
                  \itemi Подпункт 2-1
              \end{itemize}
        \item \textbf{Пункт 3}
              \begin{itemize}
                  \itemi Подпункт 3-1
                  \itemi Подпункт 3-2
              \end{itemize}
        \item \textbf{Пункт 4}
              \begin{itemize}
                  \itemi Подпункт 4-1
              \end{itemize}
        \item \textbf{Пункт 5}
              \begin{itemize}
                  \itemi Подпункт 5-1
                  \itemi Подпункт 5-2
                  \itemi Подпункт 5-3
              \end{itemize}
    \end{itemize}
\end{frame}

\begin{frame}
    \frametitle{Четыре изображения}
    \centering
    \includegraphics[width=0.35\linewidth,angle=35]{latex}
    \includegraphics[width=0.35\linewidth,angle=135]{latex}\\
    \includegraphics[width=0.35\linewidth,angle=15]{latex}
    \includegraphics[width=0.35\linewidth,angle=-15]{latex}
\end{frame}

